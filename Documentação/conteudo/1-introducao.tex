O objetivo do trabalho prático é a implementação de um sistema de telefonia simplificado, considerando elementos como clientes, telefones celulares de cada cliente, planos projetados pela operadora e ligações de dados ou de voz. Técnicas abordadas em sala de aula como polimorfismo, herança, composição e tratamento de exceções foram utilizadas na implementação, possibilitando maior abstração do problema e complexidade do sistema final. Para facilitar a entrada e saída de dados no terminal, a biblioteca \textit{ncurses} foi utilizada na classe de Interface. A implementação foi feita na linguagem C++. 

\subsection{Informações de implementação e execução}
Os arquivos de cabeçalho estão separados no diretório \textit{Headers}, os arquivos fonte no diretório \textit{Source}. Uma pequena lista de comandos foi gerada no arquivo \textit{README.md} para facilitar o entendimento do sistema e familiarização inicial. A chamada da interface está localizada no elemento \textit{main.cpp}. 

\begin{itemize}
	\item \textbf{Sistema operacional utilizado:} Manjaro Linux Illyria 18.0.4
	\item \textbf{Compilador:} gcc 8.3.0 
	\item \textbf{IDE:} Visual Studio Code 1.35.1
\end{itemize}

Para compilar e executar o código ao mesmo tempo, é necessário abrir um terminal de comando dentro do diretório \textit{Source} e digitar o seguinte comando:

\texttt{g++ *.cpp -lncurses -o main \&\& ./main}

