
O presente relatório apresentou o processo de modelagem e controle da planta \textit{TRMS}, em que o grupo já havia visitado previamente em uma prática com um grau de liberdade, apenas o rotor de arfagem estava livre, mas agora com dois graus de liberdade. O segundo grau de liberdade acrescenta um efeito de acoplamento, além das características não-lineares de cada rotor, tornando um desafio o projeto de controlador para o sistema.

A técnica de modelagem "caixa-preta" foi a escolhida pelo grupo e tanto análise da planta quanto o projeto do controlador foram feitos no domínio de Laplace, utilizando Funções de Transferência. O grupo também não considerou uma modelagem mais complexa do acoplamento dos motores, apenas a banda passante do sistema final de cada rotor para a rejeição de distúrbios que provém de uma faixa de frequência possível do acoplamento. 

Considerando a complexidade do problema e as limitações das escolhas de projeto, além das simplificações e erros de modelagem, o resultado final obtido pelos integrantes foi satisfatório e o objetivo da prática de rastreamento de referência e rejeição de distúrbio foi atendido completamente. A dinâmica da cauda apresentou resultados com erro percentual baixo, grande rejeição à distúrbios e baixa sensibilidade aos efeitos de acoplamento. O controle da arfagem foi satisfatório e também conseguiu cumprir os requisitos de projeto, entretanto, melhorias podem ser feitas em um projeto com técnicas de modelagem e controle mais recentes.

Possíveis abordagens para melhorar o desempenho do projeto são a modelagem não-linear, modelagem do acoplamento e tentativa de compensação do mesmo ou então o projeto por espaço de estados, com um sistema MIMO(\textit{Multiple Inputs Multiple Outputs}) seriam metodologias promissoras.
