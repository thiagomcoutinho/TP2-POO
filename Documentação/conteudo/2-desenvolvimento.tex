Durante a implementação do trabalho, conceitos desenvolvidos previamente como sobrecarga, uso da \textit{key const}, passagem de argumentos por referência e variáveis \textit{static} foram utilizados.

Além das classes que compõem os elementos de uma operadora de telefonia, como clientes, ligações e planos, foi necessário desenvolver classes para manipular exceções, datas e gerenciar a interação entre classes. A última tarefa citada é implementada através da classe \textbf{Interface}, responsável por criar processos acessíveis ao cliente do aplicativo; é a classe mais complexa e será descrita por último para facilitar o seu entendimento.

Essa seção contém os cabeçalhos de cada classe implementada no trabalho com uma breve explicação. Para maiores detalhes, consulte o código fonte.

\subsection{Classe Cliente} \label{sec:cliente}

\begin{lstlisting}[basicstyle=\tiny]
#ifndef CLIENTE
#define CLIENTE

#include "./Celular.h"
#include "./PrePago.h"
#include "./PosPago.h"
#include "./Date.h"

#include<vector>
#include<string>

using namespace std;

class Cliente{

	private:
		const string CPF;
		const string nome;
		string endereco;
		vector<Celular*> celulares;
	
	public:
		Cliente();
		Cliente(const string& _cpf, const string& _nome, const string& _endereco) : CPF(_cpf), nome(_nome), endereco(_endereco) {};
		~Cliente();
		
		// FUNCOES DE ADICIONAR CELULAR
		void addCelular(const string& nomePlano, const double& vlrMinuto, const double& _franquia, const double& _velAlem, \
		const double& _veloc, const Date& vencimento);
		void addCelular(const string& nomePlano, const double& vlrMinuto, const double& _franquia, const double& _velAlem, \
		const double& _veloc, Date& curr_date, const double& credito);
		
		// FUNCOES DE EFETUAR LIGACAO
		void efetuarLigacao(const int& celularIndex, const Date& timestamp, const double& duracao, const double& numTel);
		void efetuarLigacao(const int& celularIndex, const Date& timestamp, const double& duracao, const tipoDados& data_type);
		
		// FUNCOES GET
		inline string getNome()                const {return(nome);};
		inline string getCPF()                 const {return(CPF);};
		inline string getEndereco()            const {return(endereco);};
		inline vector<Celular*> getCelulares() const {return(celulares);};
};
#endif
\end{lstlisting}

A classe cliente armazena dados de CPF, nome do cliente e seu endereço, além de um vetor contendo os celulares atribuídos a ele. O código utiliza composição para incluir a classe \textit{Celular}, que também é composta por \textit{Cliente}; como essa declaração é recursiva, é necessário o uso de \textit{forward declaration}, que será abordado na classe do Celular. 

Os métodos presentes são simples, existem funções que adicionam objetos de celulares e funções que efetuam ligações a partir de um objeto celular. Ambas possuem sobrecarga em sua assinatura e utilizam alocação de memória para criar os objetos. As sobrecargas são feitas nas duas funções para diferenciar operações de criar celulares com planos do tipo Pré-pago e Pós-Pago, ou fazer ligações do tipo de Dados ou Simples.


\subsection{Classe Celular} \label{sec:celular}

\begin{lstlisting}[basicstyle=\tiny]
#ifndef CELULAR
#define CELULAR

#include "./Plano.h"
#include "./PosPago.h"
#include "./Excecao.h"
#include "./Ligacao.h"
#include "./LigacaoDados.h"
#include "./LigacaoSimples.h"
#include "./Date.h"

#include<vector>

using namespace std;

class Cliente; // Instanciando classe sem declaração completa (FORWARD DECLARATION)

class Celular{

	private:
		double numero;
		Cliente* cliente;
		Plano* plano;
		vector<Ligacao*> ligacoes;
		static double proxNumCelular;
	
	public:
		Celular();
		Celular(Cliente* c, Plano* p);
		~Celular();
		
		// FUNCOES DE LIGAÇÃO
		void ligar(const Date& timestamp, const double& duracao, const double& numTel);
		bool ligar(const Date& timestamp, const double& duracao, const tipoDados& td);
		
		// FUNÇÕES GET
		inline static double getProxNumCelular()       {return(proxNumCelular);}; // STATIC FUNCTION
		inline vector<Ligacao*> getLigacoes()    const {return(ligacoes);};
		inline double getNumero()                const {return(numero);};
		inline Plano* getPlano()                 const {return(plano);};
		inline Cliente* getCliente()             const {return(cliente);};
		
		// FUNÇÕES SET
		inline static void incrementProxNumCelular() {proxNumCelular++;};
};
#endif
\end{lstlisting}

A definição dessa classe tem caráter recursivo com a classe Cliente e portanto é necessário o uso de \textit{forward declaration}, já citado na Seção \ref{sec:cliente}. A classe deve ser declarada sem corpo, avisando ao compilador que ela existe, mas seu corpo ainda deve ser declarado em no respectivo arquivo \texttt{.cpp}; o objeto da classe ainda não declarada deve ser instanciado como ponteiro.

Cada celular possui um número único, que é gerenciado através da variável \texttt{static proxNumCelular} e armazenada na variável \texttt{double numero}. Um celular também deve possuir um plano, pertencer a apenas um cliente e conter uma lista de ligações efetuadas por ele. Para utilizar conceitos de herança e polimorfismo, os objetos da \textit{Ligações} são acessados por meio de ponteiros. 

Métodos \textit{getters} e \textit{setters} foram criados para retornar e modificar valores privados da classe. As funções sobrecarregadas \textit{ligar} podem ser chamadas tanto através da classe Cliente quanto Celular, a sobrecarga é feita para diferenciar ligações do tipo de Dados ou Simples. Essas funções desempenham a principal funcionalidade da seção atual, alguns pontos importantes são detalhados abaixo:

\begin{lstlisting}
// LIGACAO SIMPLES
void Celular::ligar(const Date& timestamp, const double& duracao, const double& numTel){
	double custo;
	custo = duracao*plano->getValorMinuto(); // CALCULA O CUSTO DA LIGACAO
	
	PosPago* ptr_plano = dynamic_cast<PosPago*>(plano); // VERIFICA QUAL TIPO DE PLANO
	if( ptr_plano == nullptr){ 
		plano->verificaData(timestamp); // VERIFICA VALIDADE DOS CREDITOS
		plano->verificaCredito(custo);  // VERIFICA CREDITO CELULAR PRE PAGO
	}
	plano->cobraCusto(custo); // COBRA CUSTO
	
	Ligacao* l = new LigacaoSimples(timestamp, duracao, custo, numTel); // LIGA
	ligacoes.push_back(l);
}
\end{lstlisting}

Foi utilizado \textit{dynamic cast} para verificar qual era o Plano do celular, em termos de código, qual classe derivada de Plano o ponteiro apontava. Para o caso de Ligação Simples com Plano Pré-Pago, a validade dos créditos tem que ser verificada e caso inválido uma exceção será jogada.

Para Ligações do tipo Dados tem que ser verificado se o Plano possui franquia, caso ele não tenha, uma exceção deve ser mostrada na tela. Para o cálculo do custo da ligação, dois fatores devem ser considerados: se a ligação foi do tipo \texttt{download} ou \texttt{upload} e se a franquia de dados foi estourada durante a ligação. 

Para ligações do tipo \texttt{upload}, a velocidade de uso será um décimo da velocidade padrão especificada no plano(velocidade de \text{download}); para o caso de limite de franquia excedido durante a ligação, a velocidade utilizada será composta pela velocidade da franquia e a velocidade após seu consumo total. Ambos os casos influenciam no custo final da ligação e foram considerados na implementação do código.


\subsection{Classe Date} \label{sec:date}

\begin{lstlisting}
#ifndef DATE
#define DATE

#include "./Excecao.h"

#include<string>
#include<vector>

using namespace std;

class Date{
	
	private:
		int dia;
		int mes;
		int ano;
	
	public:
		Date();
		Date(const int& _ano, const int& _mes, const int& _dia);
		~Date();
	
		// OPERADORES
		bool operator >  (const Date& b) const;
		bool operator >= (const Date& b) const;
		bool operator <= (const Date& b) const;
		Date & operator = (const string& str_date);
		Date & operator = (const Date& data);
		
		// FUNCOES SET
		void acrescentaTempo();
		
		// FUNCOES GET
		string convertDateToString() const;
		vector<int> getLimitesMes() const;
		inline int getDia()  const {return(dia);};
		inline int getMes()  const {return(mes);};
		inline int getAno()  const {return(ano);};
};
#endif
\end{lstlisting}

\subsection{Classe Interface} \label{sec:interface}

\subsection{Classe Exceção} \label{sec:excecao}

\subsection{Classes Base} \label{sec:classes_base}

\subsubsection{Plano} \label{sec:plano}

\subsubsection{Ligacao} \label{sec:ligacao}

\subsection{Classes Derivadas} \label{sec:classes_derivadas}

\subsubsection{PosPago : Plano} \label{sec:pospago}

\subsubsection{PrePago : Plano} \label{sec:prepago}

\subsubsection{LigacaoDados: Ligacao} \label{sec:ligacaodados}

\subsubsection{LigacaoSimples: Ligacao} \label{sec:ligacaosimples}

\subsection{Observações} \label{sec:observacoes}

TO-DO: Ligação com crédito excedido, computar ligação até o momento em que o crédito acaba.



% Dynamic casting, upcasting