Durante a implementação do trabalho, conceitos desenvolvidos previamente como sobrecarga, uso da \textit{key const}, passagem de argumentos por referência e variáveis \textit{static} foram utilizados.

Além das classes que compõem os elementos de uma operadora de telefonia, como clientes, ligações e planos, foi necessário desenvolver classes para manipular exceções, datas e gerenciar a interação entre classes. A última tarefa citada é implementada através da classe \textbf{Interface}, responsável por criar processos acessíveis ao cliente do aplicativo; é a classe mais complexa e será descrita por último para facilitar o seu entendimento.

\subsection{Classe Cliente}





\subsection{Classe Celular}

\subsection{Classe Date}

\subsection{Classe Interface}

\subsection{Classe Exceção}

\subsection{Classes Base}

\subsubsection{Plano}

\subsubsection{Ligacao}

\subsection{Classes Derivadas}

\subsubsection{PosPago : Plano}

\subsubsection{PrePago : Plano}

\subsubsection{LigacaoDados: Ligacao}

\subsubsection{LigacaoSimples: Ligacao}


% Dynamic casting, upcasting